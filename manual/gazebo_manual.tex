\documentclass[11pt, twoside, a4paper]{report}
\usepackage{dirtree} %for the tree enhierachy representation
\usepackage[english]{babel} %english dictionary
\usepackage{amsmath,amssymb,calc,bbding} %math tools
\usepackage{pdfpages} %including PDF pages
\usepackage{caption} %figures and equations' captions
\usepackage{epstopdf} %for eps figures
\usepackage{url}

\usepackage{xcolor}
\usepackage{listings}
\lstset{basicstyle=\ttfamily,
	showstringspaces=false,
	commentstyle=\color{red},
	keywordstyle=\color{blue}
}

\definecolor{gray}{rgb}{0.4,0.4,0.4}
\definecolor{darkblue}{rgb}{0.0,0.0,0.6}
\definecolor{cyan}{rgb}{0.0,0.6,0.6}
\definecolor{Maroon}{rgb}{0.5,0,0}
\definecolor{DarkOliveGreen}{rgb}{0,0.5,0}
\definecolor{olive}{rgb}{0,0.5,0}

% Title Page
\title{Manual of Gazebo-SpaceDyn Simulation}
\author{Jacob Hernandez}
\begin{document}
\maketitle

\newpage

\begin{abstract}
\end{abstract}

\chapter{Introduction}

This document intends to serve as a brief description and manual of the tools and routines developed within the context of Master Thesis titled: \textbf{Development of a Simulation Testbed for Evaluation of Impact/Contact Dynamics and Control of Non Cooperative Tumbling Satellites} in the Space Robotics Laboratory in Tohoku University in Sendai.

\chapter{Installation of the software and setup}

The main software that need to be installed is:

\begin{itemize}
	\item ROS Kinetic Kame (\url{http://wiki.ros.org/kinetic/Installation/Ubuntu})
	\item Gazebo 7 (\url{http://gazebosim.org/blog/gazebo7})
	\item Ubuntu 16 (\url{http://releases.ubuntu.com/16.04/})
	\item GNU Scientific Library (sudo apt-get install libgsl-dev)
\end{itemize}

Please follow the instructions of installation of each of them. 

Create a new catkin space:
	
	\begin{lstlisting}*[language=bash, caption={catkin space}]
	
	$ source /opt/ros/kinetic/setup.bash
	
	$ mkdir -p ~/catkin_ws/src
	$ cd ~/catkin_ws/
	$ catkin_make		
	
	\end{lstlisting}

	


\chapter{Organization of Packages in the Catkin\_Workspace}\label{catkinws}
	
	The following is an overall overview of the packages and meta-packages created in the catkin workspace to have an idea of the structure when referring to the different components of the project.  
	
	\vspace{30mm}
	
	\dirtree{%
		.1 catkin\_ws.
		.2 src.
		.3 space\_robot\_vis.
		.3 spacedyn\_ros.
		.3 test\_gazebo.
		.3 tohoku\_space\_manipulator.
		.4 active\_debris\_control.
		.4 chaser\_bringup.
		.4 chaser\_control.
		.4 chaser\_description.
		.4 spacedyn\_integration.
		.4 target\_description.
		.3 test\_spd.
	}
	%
	%}

\section{Package: space\_robot\_vis }

This package had the purpose to serve as a visualization interface between the results obtained from SpaceDyn Matlab in .dat and Gazebo Robotics Simulator.
Hence, it is important to note the following implications:
	
	\begin{enumerate}
		\item The physics engine is de-activated.
		\item Hence, the inertial parameters in the description of the robot are not relevant. 
		\item The only thing that matters is the geometry and making sure that the joints' type and geometries avoid self-collision.  
		\item 
	\end{enumerate}

%\subsection{subsection}
%
%1. Make sure that Gazebo and ROS are properlly installed.
%
%2. Create a catkin_ws ( catkin workspace ) by executing
%\$ source /opt/ros/kinetic/setup.sh
%\$ mkdir -p ~/catkin_ws
%\$ catkin_init_workspace
%\$ cd ~/catkin_ws/
%\$ catkin_make
%3. Overlay the setup.sh of the new workspace to the environment
%\$ source devel/setup.sh
%
%4. Make sure that the $ROS_PACKAGE_APTH is correct
%
%\$echo \
%ni17917
%ni
%179317943	r1794444$ROS_PACKAGE_PATH
%/home/youruser/catkin_ws/src:/opt/ros/kinetic/share:/opt/ros/kinetic/stacks
%
%5. Copy the space robot folder to ~/catkin_ws/src/
%
%5. Prepare the environment for your new space_robot_vis package using /devel/setup.sh
%$ . ~/catkin_ws/src/space_robot_vis/devel/setup.sh
%
%--------------------------To execute the visualization----------------------
%
%a. Execute the initialization file
%$ roscd space_robot_vis_vis_vis/devel
%$ . init.sh
%
%b. Follow the instructions in the command prompt.
%
%c. Please remember to disable the physics in Gazebo!
%In the left menu, look for World tab-> Physics -> enable_physics -> tick [] False
%
%d. If you want to execute again the visualization type 
%$ . ~/catkin_ws/src/space_robot_vis/devel/pub.sh
%

	

\subsection{URDF description}



\section{Package: spacedyn\_ros }
\subsection{New functions}

\section{Package: test\_gazebo}

\section{Metapackage: tohoku\_space\_manipulator}

\subsection{Package: active\_debris\_control }
\subsection{Package: chaser\_bringup}
\subsection{Package: chaser\_control}
\subsection{Package: chaser\_description}
\subsection{Package: spacedyn\_integration}
\subsection{Package: target\_description}
\section{test\_spd}

\subsection{}

Please consider that this program was developed using gazebo-7


List of things to cover in this document:


 - Reference to the thesis
 - List of plugins for the robot
 - Important considerations in the 



\end{document}          
